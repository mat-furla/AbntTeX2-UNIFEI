% Classe do documento
% -----------------------------------------------------
\documentclass[12pt,oneside,a4paper,english,brazil]{abntex2}
% -----------------------------------------------------

% -----------------------------------------------------
% Pacotes básicos 
% -----------------------------------------------------	
\usepackage{lmodern}			% Fonte Latin Modern
%\usepackage{times}				% Fonte Times New Roman
\usepackage[T1]{fontenc}		% Seleção de codigos de fonte.
\usepackage[utf8]{inputenc}		% Codificacao do documento (conversão automática dos acentos)
\usepackage{indentfirst}		% Indenta o primeiro parágrafo de cada seção
\usepackage{color}				% Controle das cores
\usepackage{graphicx}			% Inclusão de gráficos
\usepackage{microtype} 			% Para melhorias de justificação
\usepackage{enumitem}           % Para realizar enumerações com ítens
\frenchspacing					% Retira espaço extra obsoleto entre as frases.
\usepackage{unifei} 			% Modificação do abntex2 para unifei
% -----------------------------------------------------	

% -----------------------------------------------------
% Pacotes adicionais
% -----------------------------------------------------	
\usepackage{lipsum}				% para geração de dummy text
% -----------------------------------------------------	

% -----------------------------------------------------
% Configurações gerais do PDF gerado (cor dos links, etc)
% Parte destes dados são gerados automaticamente.
% -----------------------------------------------------	
\makeatletter
\hypersetup{
	%pagebackref=true,
	pdftitle={\@title}, 
	pdfauthor={\@author},
	pdfsubject={\imprimirpreambulo},
	pdfcreator={Zé da Silva},
	pdfkeywords={abntex2}{unifei}{latex}{texto}, 
	colorlinks=true,    % Links coloridos. Falso para criar caixas ao redor
	linkcolor=black,    % Cor de links internos (seções, etc)
	citecolor=black,    % Cor de links para a bibliografia
	filecolor=magenta,  % Cor de links para arquivos
	urlcolor=blue,      % Cor de links para URLs da web
	bookmarksdepth=4
}
\makeatother
% -----------------------------------------------------	

% -----------------------------------------------------
% Informações da capa e folha de rosto
% -----------------------------------------------------	
\instituicao{Universidade Federal de Itajubá - unifei}
\departamento{Instituto de Engenharia de Sistemas e Tecnologia da Informação - IESTI}
\autor{Autor - 0000000000}
\titulo{Modelo \LaTeX\ para UNIFEI}
\subtitulo{Modificação do pacote \abnTeX}
\data{ANO}
\local{Itajubá}
\preambulo{Modelo de trabalho científico realizado em \LaTeX\ contendo
		   modificações para a classe \abnTeX\ de modo a adequar as regras
	   	   da Universidade Federal de Itajubá - UNIFEI}
\orientador{Prof. Dr. Nome do Professor}
% -----------------------------------------------------	

% -----------------------------------------------------
% Compila o índice
% -----------------------------------------------------	
\makeindex