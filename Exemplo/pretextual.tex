\pretextual
% -----------------------------------------------------
% Capa e Folha de Rosto
% -----------------------------------------------------
\imprimircapa 							
\imprimirfolhaderosto						
% -----------------------------------------------------

% -----------------------------------------------------
% Dedicatória
% -----------------------------------------------------
%\begin{dedicatoria}			
%	\null
%	\vfill
%	Espaço reservado à dedicatória, elemento opcional destinado a 
%	homenagear pessoas importantes na vida do autor do trabalho.
%\end{dedicatoria}
% -----------------------------------------------------

% -----------------------------------------------------
% Agradecimentos
% -----------------------------------------------------
%\begin{agradecimentos}[Agradecimentos] 
%	Elemento opcional que é utilizado para agradecer a pessoas 
%	ou instituições que contribuíram com a realização do trabalho.
%\end{agradecimentos}

% -----------------------------------------------------
% Epígrafe
% -----------------------------------------------------
%\begin{epigrafe}
%	\null
%	\vfill
%	Espaço reservado à epígrafe, elemento opcional, elaborado 
%	conforme a ABNT NBR 10520, em que se transcreve uma citação literal,
%	com autoria, referente ao assunto abordado no trabalho.
%\end{epigrafe}
% -----------------------------------------------------

% -----------------------------------------------------
% Resumo na língua vernácula
% -----------------------------------------------------
\begin{resumo}
  Segundo a  o resumo deve ressaltar o
  objetivo, método, os resultados e as conclusões do documento. A ordem e a extensão
  destes itens dependem do tipo de resumo (informativo ou indicativo) e do
  tratamento que cada item recebe no documento original. O resumo deve ser
  precedido da referência do documento, com exceção do resumo inserido no
  próprio documento. As palavras-chave devem figurar logo abaixo do
  resumo, antecedidas da expressão Palavras-chave:, separadas entre si por
  ponto e finalizadas também por ponto.\\

  \textbf{Palavras-chaves}: latex. abntex. editoração de texto.
\end{resumo}
% -----------------------------------------------------

% -----------------------------------------------------
% Resumo na língua estrangeira
% -----------------------------------------------------
%\begin{resumo}[Abstract]
%	Apresentação, em inglês, do resumo que apresente o conteúdo de todo o 
%	trabalho (tema central, objetivo da pesquisa, aporte teórico, 
%	metodologia empregada, resultados e conclusões).\\
	
%	\textbf{Keywords:} 3 a 5 palavras-chave. Separadas entre si por ponto. Todas em inglês.
%\end{resumo}
% -----------------------------------------------------

% -----------------------------------------------------
% Lista de ilustrações
% -----------------------------------------------------
%\pdfbookmark[0]{\listfigurename}{lof}	
%\listoffigures*
%\cleardoublepage
% -----------------------------------------------------

% -----------------------------------------------------
% Lista de tabelas
% -----------------------------------------------------
%\pdfbookmark[0]{\listtablename}{lot}	
%\listoftables*
%\cleardoublepage
% -----------------------------------------------------

% -----------------------------------------------------
% Lista de abreviaturas e siglas
% -----------------------------------------------------
%\begin{siglas}
%	\item[ABNT] Associação Brasileira de Normas Técnicas
%	\item[abnTeX] ABsurdas Normas para TeX
%\end{siglas}
% -----------------------------------------------------

% -----------------------------------------------------
% Lista de símbolos
% -----------------------------------------------------
%\begin{simbolos}
%	\item[$ \Gamma $] Letra grega Gama
%	\item[$ \Lambda $] Lambda
%	\item[$ \zeta $] Letra grega minúscula zeta
%	\item[$ \in $] Pertence
%\end{simbolos}
% -----------------------------------------------------

% -----------------------------------------------------
% Sumário
% -----------------------------------------------------
\pdfbookmark[0]{\contentsname}{toc}
\tableofcontents*
\cleardoublepage
% -----------------------------------------------------