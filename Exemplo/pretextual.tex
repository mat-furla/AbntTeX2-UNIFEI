\pretextual
% -----------------------------------------------------------------------------
% Retire o '%' das seções desejadas para utilizá-las
% -----------------------------------------------------------------------------


% -----------------------------------------------------------------------------
% Capa e folha de rosto
% -----------------------------------------------------------------------------
\imprimircapa 							
\imprimirfolhaderosto						
% -----------------------------------------------------------------------------


% -----------------------------------------------------------------------------
% Dedicatória
% -----------------------------------------------------------------------------
%\begin{dedicatoria}			
%	\null
%	\vfill
%	Espaço reservado à dedicatória, elemento opcional destinado a 
%	homenagear pessoas importantes na vida do autor do trabalho.
%\end{dedicatoria}
% -----------------------------------------------------------------------------


% -----------------------------------------------------------------------------
% Agradecimentos
% -----------------------------------------------------------------------------
%\begin{agradecimentos}[Agradecimentos] 
%	Elemento opcional que é utilizado para agradecer a pessoas 
%	ou instituições que contribuíram com a realização do trabalho.
%\end{agradecimentos}
% -----------------------------------------------------------------------------


% -----------------------------------------------------------------------------
% Epígrafe
% -----------------------------------------------------------------------------
%\begin{epigrafe}
%	\null
%	\vfill
%	Espaço reservado à epígrafe, elemento opcional, elaborado 
%	conforme a ABNT NBR 10520, em que se transcreve uma citação literal,
%	com autoria, referente ao assunto abordado no trabalho.
%\end{epigrafe}
% -----------------------------------------------------------------------------


% -----------------------------------------------------------------------------
% Resumo na língua vernácula
% -----------------------------------------------------------------------------
\begin{resumo}
O seguinte relatório tratará sobre o experimento, envolvendo capacitores 
e suas propriedades, conduzido na disciplina de física experimental 3.
 O ensaio realizado tinha como objetivo, obter os valores de tensão por
tempo de carga e descarga. Para isso, foi utilizado o laboratório de 
física LDF 2, da Universidade Federal de Itajubá, e seus equipamentos, 
como os capacitores supracitados, uma placa de conexão com linhas de 
circuitos impressas, multímetro digital e outros.\\

  \textbf{Palavras-chaves}: capacitores. carga. descarga

\end{resumo}
% -----------------------------------------------------------------------------


% -----------------------------------------------------------------------------
% Resumo na língua estrangeira
% -----------------------------------------------------------------------------
%\begin{resumo}[Abstract]
%	Apresentação, em inglês, do resumo que apresente o conteúdo de todo o 
%	trabalho (tema central, objetivo da pesquisa, aporte teórico, 
%	metodologia empregada, resultados e conclusões).\\
	
%	\textbf{Keywords:} 3 a 5 palavras-chave. Separadas entre si por ponto. Todas em inglês.
%\end{resumo}
% -----------------------------------------------------------------------------------------------------


% -----------------------------------------------------------------------------
% Lista de ilustrações
% -----------------------------------------------------------------------------
%\pdfbookmark[0]{\listfigurename}{lof}	
%\listoffigures*
%\cleardoublepage
% -----------------------------------------------------------------------------


% -----------------------------------------------------------------------------
% Lista de tabelas
% -----------------------------------------------------------------------------
%\pdfbookmark[0]{\listtablename}{lot}	
%\listoftables*
%\cleardoublepage
% -----------------------------------------------------------------------------


% -----------------------------------------------------------------------------
% Lista de abreviaturas e siglas
% -----------------------------------------------------------------------------
%\begin{siglas}
%	\item[ABNT] Associação Brasileira de Normas Técnicas
%	\item[abnTeX] ABsurdas Normas para TeX
%\end{siglas}
% -----------------------------------------------------------------------------


% -----------------------------------------------------------------------------
% Lista de símbolos
% -----------------------------------------------------------------------------
%\begin{simbolos}
%	\item[$ \Gamma $] Letra grega Gama
%	\item[$ \Lambda $] Lambda
%	\item[$ \zeta $] Letra grega minúscula zeta
%	\item[$ \in $] Pertence
%\end{simbolos}
% -----------------------------------------------------------------------------


% -----------------------------------------------------------------------------
% Sumário
% -----------------------------------------------------------------------------
\pdfbookmark[0]{\contentsname}{toc}
\tableofcontents*
\cleardoublepage
% -----------------------------------------------------------------------------